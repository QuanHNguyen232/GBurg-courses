\documentclass[10pt]{article}
\pagestyle{plain} \setlength{\textwidth}{12cm}
\setlength{\oddsidemargin}{2cm} \setlength{\evensidemargin}{2cm}
\setlength{\textheight}{23cm}
\usepackage{amssymb,latexsym,amsmath,amsthm,verbatim,calc}
\usepackage{graphicx,epsfig,epstopdf,amssymb,color,makeidx,float,caption,sidecap}

\begin{document}
\title{Math 212 written homework assignment week 1}
\author{Quan Nguyen}
\maketitle

\bigskip
\noindent
{\bf Acknowledgements and sources of help}:
\begin{itemize}
    \item I asked for help from Professor Benjamin Kennedy.
\end{itemize}

\bigskip
\noindent

\section*{Problem 1}

\noindent The word “nonzero” necessary in elementary row operations.

\noindent We have a theorem that during the elementary row operations to the augmented matrix of a linear results in an equivalent system

\noindent For example, the system we have:
\begin{align*}
    x_1 + x_2 &= 1 \\
    x_1 + 2x_2 &= 3
\end{align*}
\noindent The solution set of this system is (-1, 2)

\noindent If we multiply the first row by $0$, the system will be like this:
\begin{align*}
    0 + 0 &= 0 \\
    x_1 + 2x_2 &= 3
\end{align*}
\noindent Therefore:
\begin{align*}
    x_1 + 2x_2 &= 3 \\
    x_2 &\text{ is free}
\end{align*}

\noindent In this system, we have $x_2$ is free, so there are infinite number of solutions. Here is one of the solutions it can have: (3, 0). The solution of this system (3, 0) is different from the solution of the first system (-1, 2).


\noindent Thus, these two systems are not equivalent which is contradict to the theorem above and the word "nonzero" is absolutely necessary.


\section*{Problem 2}
\begin{itemize}
    \item I started swimming when I was 5. Also, in the following years, I participated in many swimming competitions at city level and earned medals, so the activity or skill that I'm good at is swimming.
    \item An activity or skill that I'm not good at is playing musical instruments. But don't worry! I have guitar lesson this semester so that I can play and enjoy songs that I love.
    \item An activity or skill that I have improved is coding. I love coding, which is the reason why I decided to major in computer science and chose this as my future career, so much that last summer I spent most of the time learning new programming language. Also, computer science and math are tightly connected to each other, so in the near future, I want to improve my mathematical ability, too.
\end{itemize}

\end{document} 