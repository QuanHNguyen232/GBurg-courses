\documentclass[10pt]{article}
\pagestyle{plain} \setlength{\textwidth}{12cm}
\setlength{\oddsidemargin}{2cm} \setlength{\evensidemargin}{2cm}
\setlength{\textheight}{23cm}
\usepackage{amssymb,latexsym,amsmath,amsthm,verbatim,calc}
\usepackage{graphicx,epsfig,epstopdf,amssymb,color,makeidx,float,caption,sidecap}

\begin{document}
\title{Math 212 written homework assignment week 10}
\author{Quan Nguyen}
\maketitle

\bigskip
\noindent
{\bf Acknowledgements and sources of help}:
\begin{itemize}
    \item Professor Ben Ken
    \item Ben (PLA)
\end{itemize}

\bigskip
\noindent

\section*{Problem 1}
\noindent From using show(A.eigenvectors\_right()), I have these results:
\begin{itemize}
    \item Eigenvalues: $\frac{1}{3}$
    \item Multiplicities: 1
    \item Eigenvectors:
    \begin{equation*}
        \begin{bmatrix}
            1 \\ -\frac{1}{6} \\ -\frac{5}{3} \\ -\frac{1}{6} \\ 1
        \end{bmatrix}
    \end{equation*}
\end{itemize}


\section*{Problem 2}

\subsection*{a}
\noindent Given from the question:
\begin{itemize}
    \item  $\lambda$ is the only eigenvalue of A.
    \item The corresponding eigenspace has dimension 1.
    \item $\Vec{v}$ is an eigenvector.
\end{itemize}
\noindent From the eigenspace, we know that there exists only one eigenvector for the given $\lambda$. Therefore, $\Vec{v}$ must be the only eigenvector in the eigenspace. Thus, $\Vec{w}$ is not a eigenvector.

\subsection*{b}
\noindent If $(A-\lambda I)\Vec{w}$ is an eigenvector, we need to show that $A(A-\lambda I)\Vec{w} = constant (A- \lambda I) \Vec{w}$. \par
\noindent From the left hand side, we have:
\begin{align*}
    A(A-\lambda I)\Vec{w}
    &= A(A\Vec{w}-\lambda I\Vec{w}) \\
    &= A(c_1 \Vec{v} + c_2 \Vec{w} - \lambda \Vec{w}) \\
    &= c_1 A\Vec{v} + c_2 A\Vec{w} - \lambda A\Vec{w} \\
    &= c_1\lambda \Vec{v} + c_2(c_1\Vec{v} + c_2\Vec{w}) -\lambda (c_1\Vec{v} + c_2\Vec{w}) \\
    &= c_1c_2 \Vec{v} + c_2^2 \Vec{w} - c_2\lambda \Vec{w} \\
    &= c_2 (c_1\Vec{v} + c_2\Vec{w} - \lambda \Vec{w}) \\
    &= c_2 (A\Vec{w}- \lambda \Vec{w} ) \\
    &= c_2 (A -\lambda I)\Vec{w}
\end{align*}
\noindent We know that $c_2$ is a constant, so $c_2 (A -\lambda I)\Vec{w}$ can be written as $ constant (A -\lambda I)\Vec{w}$ \par
\noindent Therefore, we prove that $A(A-\lambda I)\Vec{w} = constant (A- \lambda I) \Vec{w}$.

\subsection*{c}
\noindent From (b), we proved that $(A-\lambda I)\Vec{w}$ is an eigenvector for $A$. \par
\noindent However, we are given that $\lambda$ is the only eigenvalue and $\Vec{v}$ is the only eigenvector since the eigenspace for $\lambda$ has dimension 1. \par
\noindent Therefore, any other eigenvectors of $A$ are multiple of $\Vec{v}$, which means that $(A-\lambda I)\Vec{w}$ is a multiplication of $\Vec{v}$.


\subsection*{d}
\begin{itemize}
    \item We are given that $B=\{\Vec{v}, \Vec{w}\}$ is the basis for $\mathbb{R}^2$, so if $\Vec{u}\in \mathbb{R}^2$, $\Vec{u}$ can be expressed as $\Vec{u} = a\Vec{v} + b\Vec{w}$ for any value $a, b$. (1)
    \item We are given that $\Vec{v}$ is the eigenvector of $A$, so we have:
    \begin{align*}
        A\Vec{v} &= \lambda \Vec{v} \\
        (A - \lambda I) \Vec{v} &= \Vec{0} \text{ (2)} \\
    \end{align*}
    \item From (b), we know that $(A-\lambda I)\Vec{w}$ is an eigenvector for $A$, so we have:
    \begin{align*}
        A(A-\lambda I)\Vec{w} &= \lambda (A-\lambda I)\Vec{w} \\
        (A\Vec{w} - \lambda \Vec{w})(A-\lambda I) &= \Vec{0} \\
        A^2 \Vec{w} - 2A\lambda \Vec{w} + \lambda^2 I \Vec{w} &= \Vec{0} \\
        (A - \lambda I)^2 \Vec{w} &= \Vec{0} \text{ (3)}
    \end{align*}
\end{itemize}

\noindent If we have $\Vec{u}\in \mathbb{R}^2$, we need to show that $(A - \lambda I)^2\Vec{u} = \Vec{0}$ \par
\noindent From the left hand side, we have:
\begin{align*}
    (A - \lambda I)^2\Vec{u}
    &= (A - \lambda I)^2 (a\Vec{v} + b\Vec{w}) \\
    &= a(A - \lambda I)^2 \Vec{v} + b(A - \lambda I)^2 \Vec{w} \\
    &= a(A-\lambda I)\Vec{0} + b\Vec{0} \\
    &= \Vec{0}
\end{align*}

\noindent In conclusion, if $\Vec{u}\in \mathbb{R}^2$ is any vector, we must have $(A - \lambda I)^2\Vec{u} = \Vec{0}$

\end{document} 