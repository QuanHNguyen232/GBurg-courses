\documentclass[10pt]{article}
\pagestyle{plain} \setlength{\textwidth}{12cm}
\setlength{\oddsidemargin}{2cm} \setlength{\evensidemargin}{2cm}
\setlength{\textheight}{23cm}
\usepackage{amssymb,latexsym,amsmath,amsthm,verbatim,calc}
\usepackage{graphicx,epsfig,epstopdf,amssymb,color,makeidx,float,caption,sidecap}

\begin{document}
\title{Math 212 take-home exam}
\author{Quan Nguyen}
\maketitle

\bigskip
\noindent
{\bf Honor Pledge:}
\par
\noindent I affirm that I have upheld the highest principles of honesty and integrity in my academic work and have not witnessed a violation of the Honor Code.
\par
\bigskip
\noindent Quan Nguyen

\bigskip
\noindent

\section*{Question 1}
\begin{equation*}
    A = 
    \begin{bmatrix}
        3 & -2  & -2 \\
        0 & 2 & -4 \\
        1 & -2 & 2 
    \end{bmatrix}
\end{equation*}
\noindent Now we need to find the RREF of $A$: \par
\noindent Replace R3 with R1-3R3 and multiplied R2 by $\frac{1}{2}$:
\begin{equation*}
    \begin{bmatrix}
        3 & -2  & -2 \\
        0 & 1 & -2 \\
        0 & 4 & -8 
    \end{bmatrix}
\end{equation*}
\noindent Multiplied R3 by $\frac{1}{4}$:
\begin{equation*}
    \begin{bmatrix}
        3 & -2  & -2 \\
        0 & 1 & -2 \\
        0 & 1 & -2 
    \end{bmatrix}
\end{equation*}
\noindent Replace R3 with R2-R3:
\begin{equation*}
    \begin{bmatrix}
        3 & -2  & -2 \\
        0 & 1 & -2 \\
        0 & 0 & 0 
    \end{bmatrix}
\end{equation*}
\noindent Replace R1 with R1+2R2:
\begin{equation*}
    \begin{bmatrix}
        3 & 0 & -6 \\
        0 & 1 & -2 \\
        0 & 0 & 0 
    \end{bmatrix}
\end{equation*}
\noindent Multiplied R1 with $\frac{1}{3}$
\begin{equation*}
    \begin{bmatrix}
        1 & 0 & -2 \\
        0 & 1 & -2 \\
        0 & 0 & 0 
    \end{bmatrix}
\end{equation*}

\subsection*{a}
We have: $det(A) = -a_{31}\cdot det(A_{31}) + a_{32}\cdot det(A_{32}) - a_{33}\cdot det(A_{33}) = 0$. \par
\noindent Since $det(A) = 0$, $A$ is not invertible.

\subsection*{b}
\noindent From RREF of $A$, we can find the basis for null of $A$:
\begin{equation*}
    \begin{bmatrix}
        1 & 0 & -2 \\
        0 & 1 & -2 \\
        0 & 0 & 0 
    \end{bmatrix}
\end{equation*}
\noindent We have a linear system:
\begin{align*}
    x_1 - 2x_3 &= 0 \\
    x_2 - 2x_3 &= 0 \\
    x_3 &\text{ free.}
\end{align*}
\noindent Therefore, we have this vector:
\begin{equation*}
    \begin{bmatrix}
        x_1 \\
        x_2 \\
        x_3
    \end{bmatrix}
    =
    \begin{bmatrix}
        2x_3 \\
        2x_3 \\
        x_3
    \end{bmatrix}
    = x_3
    \begin{bmatrix}
        2 \\
        2 \\
        1
    \end{bmatrix}
\end{equation*}
\noindent So, the basis of null $A$:
\begin{align*}
    Nul(A) = Span\left\{
    \begin{bmatrix}
        2 \\
        2 \\
        1
    \end{bmatrix}
    \right\}
\end{align*}


\subsection*{c}
\noindent Assume we have:
\begin{equation*}
    A\Vec{x} = \Vec{b}
\end{equation*}
\begin{equation*}
    \begin{bmatrix}
        3 & -2 & -2 \\
        0 & 2 & -4 \\
        1 & -2 & 2
    \end{bmatrix}
    \Vec{x}=
    \begin{bmatrix}
        2 \\
        4 \\
        -2
    \end{bmatrix}
\end{equation*}
\noindent If there is at least a solution for $\Vec{x}$, then $\Vec{b}$ is in column space of $A$:
\noindent Turn that equation into augmented matrix:
\begin{equation*}
    \begin{bmatrix}
        3 & -2 & -2 & 2 \\
        0 & 2 & -4 & 4 \\
        1 & -2 & 2 & -2
    \end{bmatrix}
\end{equation*}
\noindent Multiplied R1 by $\frac{1}{3}$:
\begin{equation*}
    \begin{bmatrix}
        1 & -\frac{2}{3} & -\frac{2}{3} & \frac{2}{3} \\
        0 & 2 & -4 & 4 \\
        1 & -2 & 2 & -2
    \end{bmatrix}
\end{equation*}
\noindent Replace R3 with R3-R1:
\begin{equation*}
    \begin{bmatrix}
        1 & -\frac{2}{3} & -\frac{2}{3} & \frac{2}{3} \\
        0 & 2 & -4 & 4 \\
        0 & -\frac{4}{3} & \frac{8}{3} & -\frac{8}{3}
    \end{bmatrix}
\end{equation*}
\noindent Multiplied R2 by $\frac{1}{2}$
\begin{equation*}
    \begin{bmatrix}
        1 & -\frac{2}{3} & -\frac{2}{3} & \frac{2}{3} \\
        0 & 1 & -2 & 2 \\
        0 & -\frac{4}{3} & \frac{8}{3} & -\frac{8}{3}
    \end{bmatrix}
\end{equation*}
\noindent Replace R3 with $\frac{4}{3}$R2 + R3:
\begin{equation*}
    \begin{bmatrix}
        1 & -\frac{2}{3} & -\frac{2}{3} & \frac{2}{3} \\
        0 & 1 & -2 & 2 \\
        0 & 0 & 0 & 0
    \end{bmatrix}
\end{equation*}
\noindent Replace R1 with $\frac{2}{3}$R2 + R1:
\begin{equation*}
    \begin{bmatrix}
        1 & 0 & -2 & 2 \\
        0 & 1 & -2 & 2 \\
        0 & 0 & 0 & 0
    \end{bmatrix}
\end{equation*}
\noindent From the augmented matrix right above, it is obvious that the it is consistent. Therefore, $\Vec{b}$ is in column space of $A$.


\section*{Question 2}
\subsection*{a}
\begin{equation*}
    A=
    \begin{bmatrix}
        1 & 0 \\
        0 & 1
    \end{bmatrix}
\end{equation*}
\begin{equation*}
    A^{-1}=
    \begin{bmatrix}
        1 & 0 \\
        0 & 1
    \end{bmatrix}
\end{equation*}


\subsection*{b}
\noindent Impossible. Because there are more columns than rows, so there can't be pivot in each column. Thus, the columns can't be independent.


\subsection*{c}
\begin{equation*}
    A=
    \begin{bmatrix}
        1 & 0 & 0 \\
        0 & 1 & 0 \\
        0 & 0 & 1 \\
        0 & 0 & 0
    \end{bmatrix}
\end{equation*}

\subsection*{d}
\begin{equation*}
    A=
    \left\{
    \begin{bmatrix}
        1 \\
        0 \\
        0
    \end{bmatrix};
    \begin{bmatrix}
        0 \\
        1 \\
        0
    \end{bmatrix};
    \begin{bmatrix}
        0 \\
        0 \\
        1
    \end{bmatrix}
    \right\}
\end{equation*}


\subsection*{e}
\noindent Impossible. The matrix for linear transformation from $\mathbb{R}^4$ to $\mathbb{R}^4$ must be a 4x4 matrix (a square matrix). If that matrix is onto, it has pivot in every row. Since that matrix is a square, it also has pivot in every column. Therefore, if that matrix is onto for a linear transformation, it also one-to-one.



\subsection*{f}
\noindent Impossible. Because $det(A^2) = det(A)\cdot det(A) = \left(det(A) \right)^2$. \par
\noindent We know $\left(det(A) \right)^2 \geq 0$, so $\left(det(A) \right)^2$ can't equal to -1.


\section*{Question 3}

\subsection*{a}
\noindent $B$ is a basis because:
\begin{itemize}
    \item 
    $
        \begin{bmatrix}
            1 \\
            1
        \end{bmatrix};
        \begin{bmatrix}
            1 \\
            -1
        \end{bmatrix}
    $ are independent.
    \item $
        \begin{bmatrix}
            1 \\
            1
        \end{bmatrix};
        \begin{bmatrix}
            1 \\
            -1
        \end{bmatrix}
    $ span in $\mathbb{R}^2$.
\end{itemize}


\subsection*{b}
\begin{align*}
    \Vec{x}
    &=
    \begin{bmatrix}
        1 & 1 \\
        1 & -1
    \end{bmatrix}
    \begin{bmatrix}
        \Vec{x}
    \end{bmatrix}_B \\
    &=
    \begin{bmatrix}
        1 & 1 \\
        1 & -1
    \end{bmatrix}
    \begin{bmatrix}
        2 \\
        5
    \end{bmatrix} \\
    &=
    \begin{bmatrix}
        2 + 5 \\
        2 - 5
    \end{bmatrix} \\
    &=
    \begin{bmatrix}
        7 \\
        -3
    \end{bmatrix}
\end{align*}


\subsection*{c}
\noindent Let $ 
    \begin{bmatrix}
        x
    \end{bmatrix}_B =
    \begin{bmatrix}
        c_1 \\
        c_2
    \end{bmatrix}
$
\noindent We have the $B$-coordinates $c_1, c_2$ of $\Vec{x}$ satisfy:
\begin{equation*}
    \begin{bmatrix}
        1 & 1 \\
        1 & -1
    \end{bmatrix}
    \begin{bmatrix}
        c_1 \\
        c_2
    \end{bmatrix} =
    \begin{bmatrix}
        2 \\
        0
    \end{bmatrix}
\end{equation*}

\noindent Turn it into augmented matrix:
\begin{equation*}
    \begin{bmatrix}
        1 & 1 & 2 \\
        1 & -1 & 0
    \end{bmatrix}
\end{equation*}
\noindent Replace R2 with R1-R2:
\begin{equation*}
    \begin{bmatrix}
        1 & 1 & 2 \\
        0 & 2 & 2
    \end{bmatrix}
\end{equation*}
\noindent Multiply R2 by $\frac{1}{2}$
\begin{equation*}
    \begin{bmatrix}
        1 & 1 & 2 \\
        0 & 1 & 1
    \end{bmatrix}
\end{equation*}
\noindent Replace R1 with R1-R2:
\begin{equation*}
    \begin{bmatrix}
        1 & 0 & 1 \\
        0 & 1 & 1
    \end{bmatrix}
\end{equation*}
\noindent Convert it into linear equation:
\begin{align*}
    c_1 = 1 \\
    c_2 = 1
\end{align*}
\noindent Therefore, $
\begin{bmatrix}
        x
\end{bmatrix}_B =
\begin{bmatrix}
    1 \\
    1
\end{bmatrix}
$.

\subsection*{d}
\begin{itemize}
    \item 
    \begin{equation*}
        A
        \begin{bmatrix}
            1 \\
            1
        \end{bmatrix}
        =
        \begin{bmatrix}
            2 & 1 \\
            1 & 2
        \end{bmatrix}
        \begin{bmatrix}
            1 \\
            1
        \end{bmatrix}
        =
        \begin{bmatrix}
            2 + 1 \\
            1 + 2
        \end{bmatrix}
        =
        \begin{bmatrix}
            3 \\
            3
        \end{bmatrix}
        = 3
        \begin{bmatrix}
            1 \\
            1
        \end{bmatrix}
    \end{equation*}
    \item
        \begin{equation*}
        A
        \begin{bmatrix}
            1 \\
            -1
        \end{bmatrix}
        =
        \begin{bmatrix}
            2 & 1 \\
            1 & 2
        \end{bmatrix}
        \begin{bmatrix}
            1 \\
            -1
        \end{bmatrix}
        =
        \begin{bmatrix}
            2-1 \\
            1-2
        \end{bmatrix}
        =
        \begin{bmatrix}
            1 \\
            -1
        \end{bmatrix}
    \end{equation*}
\end{itemize}



\subsection*{e}
\begin{align*}
    A^{10}
    \begin{bmatrix}
        2 \\
        0
    \end{bmatrix}
    &=
    A^{10}
    \left(
    \begin{bmatrix}
        1 \\
        1
    \end{bmatrix}
    +
    \begin{bmatrix}
        1 \\
        -1
    \end{bmatrix}
    \right) \\
    &=
    A^{10}
    \begin{bmatrix}
        1 \\
        1
    \end{bmatrix}
    +
    A^{10}
    \begin{bmatrix}
        1 \\
        -1
    \end{bmatrix} \\
    &=
    3^{10}
    \begin{bmatrix}
        1 \\
        1
    \end{bmatrix}
    +
    \begin{bmatrix}
        1 \\
        -1
    \end{bmatrix} \\
    &=
    59049
    \begin{bmatrix}
        1 \\
        1
    \end{bmatrix}
    +
    \begin{bmatrix}
        1 \\
        -1
    \end{bmatrix} \\
    &=
    \begin{bmatrix}
        59049 + 1 \\
        59049 - 1
    \end{bmatrix} \\
    &=
    \begin{bmatrix}
        59050 \\
        59048
    \end{bmatrix}
\end{align*}



\section*{Question 4}
\noindent $B$ is the subspace of $H$ which means that:
\begin{itemize}
    \item $\Vec{b_1}$ and $\Vec{b_2}$ are independent.
    \item $\Vec{b_1}$ and $\Vec{b_2}$ span in $\mathbb{R}^n$
\end{itemize}

\noindent We need to prove that $S = \{\Vec{b_1}+ \Vec{b_2}, \Vec{b_2} \}$ is a basis of $H$:
\begin{itemize}
    \item \noindent Prove that $Span\left\{ \Vec{b_1}+ \Vec{b_2}, \Vec{b_2} \right\} = \mathbb{R}^n$:
    \begin{itemize}
        \item We have $H$ is some subspace of $\mathbb{R}^n$. Therefore $\Vec{b_1} + \Vec{b_2}$ is also in $H$ and spans in $\mathbb{R}^n$. \par
    \end{itemize}
    \item Prove that $\Vec{b_1}+ \Vec{b_2}$ and $\Vec{b_2}$ are independent (Prove by contradiction):
    \begin{itemize}
        \item Assume that $\Vec{b_1}+ \Vec{b_2}$ and $\Vec{b_2}$ are dependent. Thus, $\Vec{b_1}+ \Vec{b_2} = k \cdot \Vec{b_2}$ for scalar $k$.
        \item Then, we have $\Vec{b_1} = (k-1) \Vec{b_2}$ which means that $\Vec{b_1}$ and $\Vec{b_2}$ are not independent. However, $\Vec{b_1}$ and $\Vec{b_2}$ must be independent since $B = \{\Vec{b_1}, \Vec{b_2} \}$ is a basis for $H$.
        \item Therefore, $\Vec{b_1}+ \Vec{b_2}$ and $\Vec{b_2}$ are independent.
    \end{itemize}
\end{itemize}

\noindent In conclusion, $S = \{\Vec{b_1}+ \Vec{b_2}, \Vec{b_2} \}$ is a basis for $H$.

% \noindent Having , we have:
% \begin{itemize}
%     \item The spanning set of $\left\{ \Vec{b_1} + \Vec{b_2}; \Vec{b_2} \right\}$ spans in $\mathbb{R}^n$
% \end{itemize}












\section*{Question 5}
\noindent I used these following commands:
\noindent Create matrix $A$: \par
A = matrix([ \par
[1, 0, 1, -1], \par
[0, -1, 1, 1], \par
[-1, 1, 0, 1], \par
[1,  1, -1, 0] ] ) \par
\noindent  View $A$ using show(A):
\begin{equation*}
    A=
    \begin{bmatrix}
        1 & 0 & 1 & -1 \\
        0 & -1 & 1 & 1 \\
        -1 & 1 & 0 & 1 \\
        1 & 1 & -1 & 0
    \end{bmatrix}
\end{equation*}

\noindent Find $A^{-1}$: show(A.inverse()):
\begin{equation*}
    A^{-1}=
    \begin{bmatrix}
        \frac{1}{5} & \frac{2}{5} & -\frac{1}{5} & \frac{3}{5} \\
        \frac{2}{5} & -\frac{1}{5} & \frac{3}{5} & \frac{1}{5} \\
        \frac{3}{5} & \frac{1}{5} & \frac{2}{5} & -\frac{1}{5} \\
        -\frac{1}{5} & \frac{3}{5} & \frac{1}{5} & \frac{2}{5}
    \end{bmatrix}
\end{equation*}


\noindent Find $A^{9}$: show(A\^{}9):
\begin{equation*}
    A^{9}=
    \begin{bmatrix}
        17 & 32 & -71 & 23 \\
        32 & -225 & 153 & 41 \\
        23 & 41 & 8 & -71 \\
        -71 & 153 & -89 & 8
    \end{bmatrix}
\end{equation*}


\end{document} 