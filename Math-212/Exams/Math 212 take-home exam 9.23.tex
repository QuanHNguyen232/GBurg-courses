\documentclass[10pt]{article}
\pagestyle{plain} \setlength{\textwidth}{12cm}
\setlength{\oddsidemargin}{2cm} \setlength{\evensidemargin}{2cm}
\setlength{\textheight}{23cm}
\usepackage{amssymb,latexsym,amsmath,amsthm,verbatim,calc}
\usepackage{graphicx,epsfig,epstopdf,amssymb,color,makeidx,float,caption,sidecap}

\begin{document}
\title{Math 212 take-home exam}
\author{Quan Nguyen}
\maketitle

\bigskip
\noindent
{\bf Honor Pledge:}
\par
\noindent I affirm that I have upheld the highest principles of honesty and integrity in my academic work and have not witnessed a violation of the Honor Code.
\par
\bigskip
\noindent Quan Nguyen

\bigskip
\noindent

\section*{Question 1}
\begin{align*}
    3x_1 - 9x_2 &= 12 \\
    x_1 - 3x_2 &= 4
\end{align*}

\subsection*{(a)}
\noindent Write the system above as an augmented matrix:
\begin{equation*}
    \begin{bmatrix}
        3 & -9 && 12 \\
        1 & -3 && 4
    \end{bmatrix}
\end{equation*}

\noindent Change R1 into $\frac{1}{3}$R1:
\begin{equation*}
    \begin{bmatrix}
        1 & -3 && 4 \\
        1 & -3 && 4
    \end{bmatrix}
\end{equation*}

\noindent Alternate R2 with R1-R2:
\begin{equation*}
    \begin{bmatrix}
        1 & -3 && 4 \\
        0 & 0 && 0
    \end{bmatrix}
\end{equation*}

\noindent Turn the augmented matrix back to the system:
\begin{align*}
    x_1 - 3x_2 &= 4 \\
    0 &= 0
\end{align*}

\noindent This means that, the solution set of this system is:
\begin{align*}
    x_1 &= 3x_2 + 4 \\
    x_2 &\text{ is free}
\end{align*}

\noindent The parametric vector form of the solution set is:
\begin{equation*}
    \begin{bmatrix}
        3x_2 + 4 \\
        x_2
    \end{bmatrix}
    =
    \begin{bmatrix}
        3x_2 \\
        x_2
    \end{bmatrix}
    +
    \begin{bmatrix}
        4 \\
        0
    \end{bmatrix}
    =
    x_2
    \begin{bmatrix}
        3 \\
        1
    \end{bmatrix}
    +
    \begin{bmatrix}
        4 \\
        0
    \end{bmatrix}
\end{equation*}


\subsection*{(b)}
\noindent From the parametric vector form above, I can easily find the solutions of the system by plugging in any values of $x_2$.
\begin{itemize}
    \item If $x_2 = 0$, the solution is $(4, 0)$.
    \item If $x_2 = 1$, the solution is $(7, 1)$.
\end{itemize}



\subsection*{(c)}
\noindent The value of $x_1$ and $x_2$ that are not a solution of this system is: $(0, 0)$.

\section*{Question 2}
\noindent The given matrix equation:
\begin{equation*}
    \begin{bmatrix}
        1 & -3 & -5 && -a \\
        2 & 4 & -2 && 1 \\
        0 & 5 & 4 && 2a
    \end{bmatrix}
    \begin{bmatrix}
        x_1 \\
        x_2 \\
        x_3
    \end{bmatrix}
    =
    \begin{bmatrix}
        -a \\
        1 \\
        2a
    \end{bmatrix}
\end{equation*}


\noindent Write the matrix equation as augmented matrix:
\begin{equation*}
    \begin{bmatrix}
        1 & -3 & -5 && -a \\
        2 & 4 & -2 && 1 \\
        0 & 5 & 4 && 2a
    \end{bmatrix}
\end{equation*}

\noindent Now I turn it into REF by these following steps:
\par
\noindent Replace R2 with $\frac{1}{2}$R2:
\begin{equation*}
    \begin{bmatrix}
        1 & -3 & -5 && -a \\
        1 & 2 & -1 && \frac{1}{2} \\
        0 & 5 & 4 && 2a
    \end{bmatrix}
\end{equation*}

\noindent Switch R2 and R1 and replace R2 with R2-R1:
\begin{equation*}
    \begin{bmatrix}
        1 & 2 & -1 && \frac{1}{2} \\
        0 & 5 & 4 && \frac{1}{2} + a \\
        0 & 5 & 4 && 2a
    \end{bmatrix}
\end{equation*}

\noindent Replace R3 with R3-R2:
\begin{equation*}
    \begin{bmatrix}
        1 & 2 & -1 && \frac{1}{2} \\
        0 & 1 & \frac{4}{5} && \frac{\frac{1}{2} + a}{5} \\
        0 & 0 & 0 && a - \frac{1}{2}
    \end{bmatrix}
\end{equation*}


\noindent Turn it back to the linear system:

\begin{align*}
    x_1 + 2x_2 - x_3 &= \frac{1}{2} \\
    2x_2 + \frac{8}{5}x_3 &= \frac{1 + 2a}{5} \\
    0 &= a - \frac{1}{2}
\end{align*}

\noindent The system above is only consistent when the third row is $0=0$: \par
\noindent So:
\begin{align*}
    a - \frac{1}{2} &= 0 \\
    a &= \frac{1}{2}
\end{align*}

\noindent In conclusion, $a = \frac{1}{2}$ will make the given matrix equation consistent.

\section*{Question 3}
\subsection*{(a)}
\begin{equation*}
    \begin{bmatrix}
        1 \\
        0
    \end{bmatrix};
    \begin{bmatrix}
        0 \\
        1
    \end{bmatrix}
\end{equation*}

\subsection*{(b)}
\noindent It is impossible because 2 vectors can only span in $R^2$ or create a plane in $R^3$.


\subsection*{(c)}
\begin{equation*}
    A=
    \begin{bmatrix}
        1 & 0 \\
        0 & 1 \\
        0 & 0
    \end{bmatrix}
\end{equation*}


\subsection*{(d)}
\begin{equation*}
    A=
    \begin{bmatrix}
        a_{1,1} & a_{1,2} & a_{1,3} \\
        a_{2,1} & a_{2,2} & a_{2,3}
    \end{bmatrix}
\end{equation*}
\noindent It is impossible because if $A$ is viewed as one-to-one linear transformation, the RREF of 2x3 matrix A has to have a pivot in each column (3 pivots). However, $A$ has only 2 rows, which means that it can has only 2 pivots.





\section*{Question 4s}
\subsection*{(a)}

\noindent False.
\noindent Counterexample:
\begin{equation*}
    A=
    \begin{bmatrix}
        1 & 0 \\
        0 & 1
    \end{bmatrix}
\end{equation*}
\noindent Columns of A is independent since there is a pivot in each column.

\begin{equation*}
    A^c=
    \begin{bmatrix}
        1 & 0
    \end{bmatrix}
\end{equation*}
\noindent Columns of A is not independent since there is only a pivot in the first column out of two columns.

\subsection*{(b)}

\noindent True.
\par
\noindent Because given columns of A, for all cases from Theorem 7, Theorem 8, and Theorem 9, the reduction of the last row does not affect these columns. For example:
\begin{itemize}
    \item Theorem 7, the column 3 is the linear combination of the other 2 columns even the last row is removed:
    \begin{equation*}
        A=
        \begin{bmatrix}
            1 & 0 & 1 \\
            0 & 1 & 1 \\
        \end{bmatrix};
        A^c=
        \begin{bmatrix}
            1 & 0 & 1 \\
        \end{bmatrix}
    \end{equation*}
    \item Theorem 8, columns of A is dependent when there are more vectors than entries, so removing the last row keeps less entries than before.
    \item Theorem 9, by removing the last row, vector $\Vec{0}$ is still $\Vec{0}$, but from $R^m$ to $R^{m-1}$
\end{itemize}


\section*{Question 5}
\noindent I used these following commands to find the solution of the system:
\begin{itemize}
    \item Create matrix A:
    \par
    A = matrix(4, 5, [\par
[-2, 1, -1, 0, -1],\par
[3, -2, 2, 4, 0],\par
[-3, 1, 1, 7, -3],\par
[2, 0, 4, 1, 1]])
    \item View that matrix: show(A)
    \item View the RREF of matrix A: show(A.rref())
\end{itemize}

\noindent The unique solution I got is:
\begin{itemize}
    \item $x_1 = -2$
    \item $x_2 = \frac{-7}{2}$
    \item $x_3 = \frac{3}{2}$
    \item $x_4 = -1$
\end{itemize}

\end{document} 