\documentclass[12pt]{article}
\usepackage{latexsym}
\usepackage{amsfonts}
\usepackage{amsmath}

%%Formatting
\topmargin -.2in
\textheight 9.2in
\evensidemargin 0in
\oddsidemargin 0in
\textwidth 6.5in
\parskip .1in

\pagestyle{empty}

\begin{document}

\hrule
\vspace{.2cm}

{\Large \noindent Math 112 Honors
\hfill
B\'ela Bajnok}

\vspace{.3cm}
\hrule

{\Large \noindent 
Exam 6
\hfill
Fall 2020}

\vspace{.3cm}
\hrule

\noindent NAME:  Quan Nguyen

\noindent \hrulefill\rule{0pt}{4pt}

\noindent Write and sign the full Honor Pledge here:

\vspace{2mm}

I affirm that I have upheld the highest principles of honesty and integrity in my academic work and have not witnessed a violation of the Honor Code. \par

Quan Nguyen

\vspace{8mm}

\noindent \hrulefill\rule{0pt}{4pt}

\noindent {\bf General instructions -- Please read!}

\begin{itemize}
  
\item The purpose of this exam is to give you an opportunity to explore a complex and challenging question, gain a fuller view of calculus and its applications, and develop some creative writing, problem solving, and research skills.

\item  {\bf All your assertions must be completely and fully justified.} At the same time, you should aim to be as concise as possible; avoid overly lengthy arguments and unnecessary components.  Your grade will be based on both mathematical accuracy and clarity of presentation.

\item 
Your finished exam should read as an article, consisting of complete sentences, thorough explanations, and exhibit correct grammar and punctuation.

\item I encourage you to prepare your exam using LaTeX.  However, you may use instead other typesetting programs that you like, and you may use hand-writing or hand-drawing for some parts of your exam.  In any case, {\bf the final version that you submit must be in PDF format.}  

\item It is acceptable (and even encouraged) to discuss the exams with other students in your class or with the PLA.  However,  {\bf  you must individually write up all parts of your exams.}

\item {\bf You may use the text, your notes, and your homework, but no other sources.}
 
\item You must write out a complete, honest, and detailed acknowledgment of all assistance you received and all resources you used (including other people) on all written work submitted for a grade.

\item Submit your exam to me by email at bbajnok@gettysburg.edu by the deadline announced in class.







\end{itemize}

\noindent {\bf Good luck!}


\newpage

\begin{center}

{\Large Some Serious Series} 

\end{center}

\vspace*{.2in}

Let us define the sequence $${\displaystyle (a_n)_{n=1}^{\infty} = \left(\frac{1}{10},-\frac{\pi^2}{100},+\frac{\pi^4}{1000},-\frac{\pi^6}{10000},+- \cdots \right)}$$ and the function $${\displaystyle f(x)=\frac{1}{10+x^2}}.$$

\begin{enumerate}
  
\item  

\begin{enumerate}

\item Find an explicit formula for $a_n$.  (Make sure that $a_1=1/10$.)




\item Find $\lim_{n \rightarrow \infty} a_n$.



\item Find the exact value of $\sum_{n=1}^{\infty} a_n$.



\end{enumerate}


\item  

\begin{enumerate}

\item Use a known series to find the infinite Maclaurin series $P_{\infty}(x)$ for $f(x)$.



\item Verify your answer to part (a) by finding the quartic (degree four) Maclaurin polynomial $P_{4}(x)$ for $f(x)$ using differentiation.

\item How is this problem related to Problem 1 above?


\end{enumerate}

\item  


\begin{enumerate}

\item Use integration rules to find ${\displaystyle \int f(x) \mathrm{d} x}.$

\item Use Problem 2 above to find ${\displaystyle \int f(x) \mathrm{d} x}.$


\item Use part (a) above to find the exact value of ${\displaystyle \int_0^{\pi} f(x) \mathrm{d} x}.$

\item Use part (b) above to approximate ${\displaystyle \int_0^{\pi} f(x) \mathrm{d} x}.$  Compare your answer to part (c).

\item Use part (a) above to find the exact value of ${\displaystyle \int_0^{\infty} f(x) \mathrm{d} x}.$  Use your calculator to verify your answer.

\item Explain why part (b) cannot be used for ${\displaystyle \int_0^{\infty} f(x) \mathrm{d} x}.$



\end{enumerate}
       
\end{enumerate}


\newpage

\begin{center}
\underline {\Large {My work} }
\end{center}


%-----Question 1----------------
\section*{Question 1:}
\subsection*{(a) Find an explicit formula for $a_n$:}
The sequence of $a$:
\begin{align*}
    a &= \left(\frac{1}{10},-\frac{\pi^2}{100},\frac{\pi^4}{1000},-\frac{\pi^6}{10000}, \cdots \right) \\
    &= \left[\frac{1}{10},
    \frac{1}{10}\cdot \left(\frac{-\pi^2}{10} \right),
    \frac{1}{10}\cdot \left(\frac{\pi^4}{100} \right),
    \frac{1}{10}\cdot \left(\frac{-\pi^6}{1000} \right),
    \cdots \right] \\
    &=\left[
    \frac{1}{10}\cdot \left(\frac{-\pi^2}{10} \right)^0,
    \frac{1}{10}\cdot \left(\frac{-\pi^2}{10} \right)^1,
    \frac{1}{10}\cdot \left(\frac{-\pi^2}{10} \right)^2,
    \frac{1}{10}\cdot \left(\frac{-\pi^2}{10} \right)^3,
    \cdots \right] \\
    &=\left[
    \frac{1}{10}\cdot \left(\frac{-\pi^2}{10} \right)^{1-1},
    \frac{1}{10}\cdot \left(\frac{-\pi^2}{10} \right)^{2-1},
    \frac{1}{10}\cdot \left(\frac{-\pi^2}{10} \right)^{3-1},
    \frac{1}{10}\cdot \left(\frac{-\pi^2}{10} \right)^{4-1},
    \cdots \right] \\
\end{align*}
Thus, the general formula for $a_n$ is:
$$(a_n)_{n=1}^{\infty} = \frac{1}{10}\cdot \left(\frac{-\pi^2}{10} \right)^{n-1}$$


\subsection*{(b) Find $\displaystyle \lim_{n \rightarrow \infty} a_n$:}
\begin{align*}
    \lim_{n\to \infty} a_n &= \lim_{n\to \infty} \frac{1}{10}\cdot \left(\frac{-\pi^2}{10} \right)^{n-1} \\
    &=\frac{1}{10} \lim_{n\to \infty} \left(\frac{-\pi^2}{10} \right)^{n-1} \\
\end{align*}

Because:
\begin{align*}
    &\pi^2 = 9.8696 < 10\\
    \Longleftrightarrow & \frac{\pi^2}{10} < 1 \\
    \Longleftrightarrow & -1<\frac{-\pi^2}{10} < 1
\end{align*}

So:
\begin{align*}
    \lim_{n\to \infty} a_n &= 
    \frac{1}{10} \lim_{n\to \infty} \left(\frac{-\pi^2}{10} \right)^{n-1} \\
    &= \frac{1}{10}\cdot 0 \\
    &= 0
\end{align*}











\subsection*{(c) Find the exact value of $\displaystyle \sum_{n=1}^{\infty} a_n$:}
\begin{align*}
    \sum_{n=1}^{\infty} a_n
    &= \sum_{n=1}^{\infty} \frac{1}{10}\cdot \left( \frac{-\pi^2}{10} \right)^{n-1} \\
    &= \frac{1}{10} + \frac{-\pi^2}{100}+ \frac{\pi^4}{1000}+ \frac{-\pi^3}{10000}+ \cdots\\
\end{align*}

The sum of $a_n$ with $n$ from $1\to \infty$ is a Geometric Series $\left(a= \frac{1}{10}, r=\frac{-\pi^2}{10}\right)$, so I can use the formula to calculate the exact value of this series:
\begin{align*}
    \sum_{n=1}^{\infty} a_n &= \frac{a}{1-r}= a \div \left(1- r \right)\\
    &= \frac{1}{10}\div \left[ 1-\left( \frac{-\pi^2}{10} \right) \right]\\
    &= \frac{1}{10}\div \left(\frac{10+ \pi^2}{10} \right)\\
    &= \frac{1}{10+\pi^2}\\
\end{align*}










%-----Question 2----------------
\newpage
\section*{Question 2:}
\subsection*{(a) Find Maclaurin Series for $f(x)$:}
Maclaurin series general formula from an function $f(x)$ at $x=a$:
$$\displaystyle \frac{f^{(n)}(a)}{n!} \left(x-a \right)^n $$


A known Maclaurin Series at $a=0$:
\begin{align*}
    \frac{1}{1-x} &= \sum_{n=0}^{\infty} x^n\\
    &= 1+ x+ x^2+ x^3+ x^4+ \dots\\
    &= \frac{1}{0! \left(1- 0 \right)}
    + \frac{\left(x- 0 \right)}{1! \left(1- 0 \right)^2}
    + \frac{2 \left(x- 0 \right)^2}{2! \left(1- 0 \right)^3}
    + \frac{6 \left(x- 0 \right)^3}{3! \left(1- 0 \right)^4}
    + \cdots
\end{align*}

If I replace $1$ in the denominator with $10$, and $x$ with $-x^2$, I will get the Maclaurin Series of function $f(x)$:
\begin{align*}
    \frac{1}{10-x^2} &= 
    \frac{1}{0! \left(10- 0 \right)}
    + \frac{\left(-x^2- 0 \right)}{1! \left(10- 0 \right)^2}
    + \frac{2 \left(-x^2- 0 \right)^2}{2! \left(10- 0 \right)^3}
    + \frac{6 \left(-x^2- 0 \right)^3}{3! \left(10- 0 \right)^4}
    + \cdots\\
    &= \frac{1}{10}
    + \frac{\left(-x^2 \right)}{10^2}
    + \frac{\left(-x^2 \right)^2}{10^3}
    + \frac{\left(-x^2 \right)^3}{10^4}
    + \cdots\\
    P(x)&= \frac{1}{10}
    - \frac{x^2}{10^2}
    + \frac{x^4}{10^3}
    - \frac{x^6}{10^4}
    + \cdots\\
\end{align*}


\subsection*{(b) Verify the answer:}

\begin{align*}
    &f(x)=\frac{1}{10+x^2}\\
    \Longrightarrow
    &f'(x)= \frac{-2x}{\left(10+x^2 \right)^2}\\
    \Longrightarrow
    &f''(x)= \frac{6x^2- 20}{\left(10+x^2 \right)^3}\\
    \Longrightarrow
    &f'''(x)= \frac{-24x\left(x^2-10 \right)}{\left(10+x^2 \right)^4}\\
    \Longrightarrow
    &f''''(x)= \frac{120x^4- 2400x^2+ 2400}{\left(10+x^2 \right)^5}
\end{align*}
*\textit{The calculation of differentiation is in the Appendix section}


\noindent The Maclaurin Series has a general formula:
$$a_0+ a_1x+ a_2x^2+ a_3x^3+ a_4x^4+ a_5x^5+ a_6x^6+ \cdots$$


\subsubsection*{With $\displaystyle f(x)=\frac{1}{10+x^2}$, and $x=0$:}

\begin{align*}
    f(0)=\frac{1}{10+0} &= a_0+ a_1x+ a_2x^2+ a_3x^3+ a_4x^4+ a_5x^5+ a_6x^6+ \cdots\\
    \Longleftrightarrow
    \frac{1}{10} &= a_0+ a_1\cdot 0+ a_2\cdot 0+ a_3\cdot 0+ a_4\cdot 0+ a_5\cdot 0+ a_6\cdot 0+ \cdots\\
    \Longleftrightarrow a_0 &= \frac{1}{10}
\end{align*}


\subsubsection*{With $\displaystyle f'(x)= \frac{-2x}{\left(10+x^2 \right)^2}$, and $x=0$:}
\begin{align*}
    f'(0)=\frac{0}{\left(10+0\right)^2} &= a_1+ 2a_2x+ 3a_3x^2+ 4a_4x^3+ 5a_5x^4+ 6a_6x^5+ \cdots\\
    0 &= a_1+ 2a_2\cdot 0+ 3a_3\cdot 0+ 4a_4\cdot 0+ 5a_5\cdot 0+ 6a_6\cdot 0+ \cdots\\
    \Longrightarrow a_1 &= 0
\end{align*}



\subsubsection*{With $\displaystyle f''(x)= \frac{6x^2- 20}{\left(10+x^2 \right)^3}$, and $x=0$:}

\begin{align*}
    f''(0)= \frac{6\cdot 0- 20}{\left(10+ 0 \right)^3}
    &= 2a_2+ 6a_3x+ 12a_4x^2+ 20a_5x^3+ 30a_6x^4+ \cdots\\
    - \frac{20}{10^3} &= 2a_2+ 6a_3\cdot 0+ 12a_4\cdot 0+ 20a_5\cdot 0+ 30a_6\cdot 0+ \cdots\\
    \Longrightarrow a_2 &=- \frac{10}{10^3}=- \frac{1}{10^2}
\end{align*}



\subsubsection*{With $\displaystyle f'''(x)= \frac{-24x\left(x^2-10 \right)}{\left(10+x^2 \right)^4}$, and $x=0$:}

\begin{align*}
    f'''(0)= \frac{-24\cdot 0\left(0- 10 \right)}{\left(10+ 0 \right)^4}
    &= 6a_3+ 24a_4x+ 60a_5x^2+ 120a_6x^3+ \cdots\\
    0 &= 6a_3+ 24a_4\cdot 0+ 60a_5\cdot 0+ 120a_6\cdot 0+ \cdots\\
    \Longrightarrow a_3 &= 0
\end{align*}


\subsubsection*{With $\displaystyle f''''(x)= \frac{120x^4- 2400x^2+ 2400}{\left(10+x^2 \right)^5}$, and $x=0$:}

\begin{align*}
    f''''(0)= \frac{120\cdot 0- 2400\cdot 0+ 2400}{\left(10+ 0 \right)^5}
    &= 24a_4+ 120a_5x+ 360a_6x^2+ \cdots\\
    \frac{2400}{10^5} &= 24a_4+ 120a_5\cdot 0+ 360a_6\cdot 0+ \cdots\\
    \Longrightarrow a_4 &= \frac{100}{10^5}= \frac{1}{10^3}
\end{align*}

\subsubsection*{Conclusion:}
\noindent From the calculations above, I have $\displaystyle a_0=\frac{1}{10},\; a_2=\frac{-1}{10^2},\; \text{and}\; a_4=\frac{1}{10^3}$. For every 2 $a$, the value is multiplied by $\displaystyle \frac{-1}{10}$, so the following values of $a$ will be: $\displaystyle a_6=\frac{-1}{10^4},\; a_8=\frac{1}{10^5},\; a_{10}=\frac{-1}{10^6},\; \dots$

\noindent Therefore, after plugging $a$ back into the Maclaurin Series, it will be:

\begin{align*}
    & a_0+ a_1x+ a_2x^2+ a_3x^3+ a_4x^4+ a_5x^5+ a_6x^6+ \cdots\\
    &= \frac{1}{10}
    + 0
    + \left(\frac{-1}{10^2} \right) x^2
    + 0
    + \left(\frac{1}{10^3} \right) x^4
    + 0
    + \left(\frac{-1}{10^4} \right) x^6
    + \cdots\\
    &= \frac{1}{10}
    + \frac{-x^2}{10^2}
    + \frac{x^4}{10^3}
    + \frac{-x^6}{10^4}
    + \cdots\\
\end{align*}

The result of using a known series to find the Maclaurin series of $f(x)$ is the same as using derivative, so the series found in part (a) is correct.


\subsection*{(c) Relation of this problem to Problem 1:}
\noindent In the Problem 1, $a_n$ is a sequence with the formula: $$\displaystyle (a_n)_{n=1}^{\infty} = \frac{1}{10}\cdot \left(\frac{-\pi^2}{10} \right)^{n-1}= 
\frac{1}{10},-\frac{\pi^2}{100},\frac{\pi^4}{1000},-\frac{\pi^6}{10000}, \cdots $$


\noindent From the Problem 2, I have $f(x)$ with $f(\pi)$ is the sum of all terms in sequence $a_n$:

\begin{align*}
    & f(x)=\frac{1}{10} \sum_{n=1}^\infty \left(- \frac{x^2}{10} \right)^{n-1}= \frac{1}{10}
    - \frac{x^2}{10^2}
    + \frac{x^4}{10^3}
    - \frac{x^6}{10^4}
    + \cdots\\
    \Longrightarrow
    & f(\pi)= \frac{1}{10}
    - \frac{\pi^2}{10^2}
    + \frac{\pi^4}{10^3}
    - \frac{\pi^6}{10^4}
    + \cdots\\
\end{align*}









    
    
%-----Question 3----------------
\newpage
\section*{Question 3:}
\subsection*{(a) Find ${\displaystyle \int f(x) \;\mathrm{d}x}$ using Integration Rules:}
\begin{align*}
    \int f(x) \;\mathrm{d}x &= \int \frac{1}{10+x^2} \;\mathrm{d}x\\
    &= \int \frac{1}{10\left(1+ \frac{x^2}{10} \right)} \;\mathrm{d}x\\
    &= \frac{1}{10} \int \frac{1}{1+ \left(\frac{x}{\sqrt{10}} \right)^2} \;\mathrm{d}x\\
    &= \frac{1}{10}\cdot \frac{\sqrt{10}}{1} \int \frac{\frac{1}{\sqrt{10}}}{1+ \left(\frac{x}{\sqrt{10}} \right)^2} \;\mathrm{d}x\\
    &= \frac{1}{\sqrt{10}}\cdot \arctan{\frac{x}{\sqrt{10}}} +C \\
\end{align*}






\subsection*{(b) Find ${\displaystyle \int f(x) \;\mathrm{d}x}$ using Problem 2:}

$$
P(x) = \frac{1}{10}
- \frac{x^2}{10^2}
+ \frac{x^4}{10^3}
- \frac{x^6}{10^4}
+ \cdots\\
$$

\begin{align*}
    \Longrightarrow
    \int P(x)\;\mathrm{d}x
    &= \int \left( \frac{1}{10}
    - \frac{x^2}{10^2}
    + \frac{x^4}{10^3}
    - \frac{x^6}{10^4}
    + \cdots \right) \;\mathrm{d}x\\
    &= \frac{x}{10}
    - \frac{x^3}{3\cdot 10^2}
    + \frac{x^5}{5\cdot 10^3}
    - \frac{x^7}{7\cdot 10^4}
    + \cdots\\
    &= \frac{x}{1\cdot 10}
    + \frac{x}{3\cdot 10} \left(\frac{-x^2}{10} \right)
    + \frac{x}{3\cdot 10} \left(\frac{-x^2}{10} \right)^2
    + \frac{x}{3\cdot 10} \left(\frac{-x^2}{10} \right)^3
    + \cdots\\
    &= \frac{x}{10} \sum_{n=0}^\infty \left(\frac{1}{2n+1}  \right) \left(\frac{-x^2}{10} \right)^n
\end{align*}











\subsection*{(c) Find the exact value of ${\displaystyle \int_0^{\pi} f(x) \;\mathrm{d}x}$ using part (a):}
\begin{align*}
    \int_0^{\pi} f(x) \;\mathrm{d}x &= \int_0^{\pi} \frac{1}{10+x^2} \;\mathrm{d}x\\
    &= \frac{1}{\sqrt{10}} \left( \arctan{\frac{x}{\sqrt{10}}} \right)_0^{\pi}\\
    &= \frac{1}{\sqrt{10}} \left( \arctan{\frac{\pi}{\sqrt{10}}}
    -\arctan{0}\right)\\
    &= \frac{1}{\sqrt{10}}\cdot \arctan{\frac{\pi}{\sqrt{10}}}
\end{align*}












\subsection*{(d) Approximate $\displaystyle \int_0^{\pi} f(x) \;\mathrm{d}x$ using part (b):}

From (b):
\begin{align*}
    \int P(x)\;\mathrm{d}x
    &= \frac{x}{10}
    - \frac{x^3}{3\cdot 10^2}
    + \frac{x^5}{5\cdot 10^3}
    - \frac{x^7}{7\cdot 10^4}
    + \frac{x^9}{9\cdot 10^5}
    - \cdots\\
    \Longrightarrow
    \int_0^{\pi} P(x) \;\mathrm{d}x
    &=\left( \frac{\pi}{10}
    - \frac{\pi^3}{3\cdot 10^2}
    + \frac{\pi^5}{5\cdot 10^3}
    - \frac{\pi^7}{7\cdot 10^4}
    + \frac{\pi^9}{9\cdot 10^5}
    - \frac{\pi^{11}}{11\cdot 10^6}
    + \cdots\right)-0\\
    & \approx 0.23524
\end{align*}














\subsection*{(e) Find the exact value of ${\displaystyle \int_0^{\infty} f(x) \;\mathrm{d}x}$ using part (a):}
\begin{align*}
    \int_0^{\infty} f(x) \;\mathrm{d}x &= \int_0^{\infty} \frac{1}{10+x^2} \;\mathrm{d}x\\
    &= \frac{1}{\sqrt{10}} \left( \arctan{\frac{x}{\sqrt{10}}} \right)_0^{\infty}\\
    &= \frac{1}{\sqrt{10}} \left( \arctan{\frac{\infty}{10}} -\arctan{0} \right)\\
    &= \frac{1}{\sqrt{10}} \left( \frac{\pi}{2} -0 \right)\\
    &= \frac{1}{\sqrt{10}}\cdot \frac{\pi}{2} \\
    &= \frac{\pi}{2\sqrt{10}}\\
\end{align*}
















\subsection*{(f) Why part (b) cannot be used for ${\displaystyle \int_0^{\infty} f(x) \;\mathrm{d}x}$:}

\noindent The function $\displaystyle P(x)=\frac{1}{10}- \frac{x^2}{10^2}+ \frac{x^4}{10^3}- \frac{x^6}{10^4}+ \cdots$ is only approximate to $\displaystyle f(x)=\frac{1}{10+ x^2}$ when $\displaystyle x\in \left(-\sqrt{10},\; \sqrt{10} \right)$. This means that when $x$ in $P(x)$ is getting closer to $\displaystyle -\sqrt{10}$ or $\displaystyle \sqrt{10}$, the value increases to infinity, and there is no value at $x= \sqrt{10}$. \par
\noindent Therefore, integral of part (b) can only be used in range $\displaystyle \left(-\sqrt{10}, \sqrt{10} \right)$. However:
\begin{align*}
    \int_0^\infty P(x) \;\mathrm{d}x
    = \int_0^{\sqrt{10}} P(x) \;\mathrm{d}x
    + \int_{\sqrt{10}}^{\infty} P(x) \;\mathrm{d}x
\end{align*}

\noindent Since the function $P(x)$ does not exist in the range from $\displaystyle \sqrt{10}$ to infinity, the integral of $P(x)$: $\displaystyle \int_{\sqrt{10}}^{\infty} P(x) \;\mathrm{d}x$ can not be calculated.


%---------Appendix----------
\newpage
\begin{center}
    \section*{The Appendix}
\end{center}
 \begin{itemize}
    \item 
    \begin{align*}
        f(x) &= \frac{1}{10+x^2} = \left(10+x^2 \right)^{-1}
    \end{align*}
    
    \item
    \begin{align*}
        f'(x) &= \left[\left(10+x^2 \right)^{-1} \right]'\\
        &= \left(-1 \right) \left(10+x^2 \right)^{-2} \left(2x \right)\\
        &= \left(-2x \right) \left(10+x^2 \right)^{-2}\\
        &= \frac{-2x}{\left(10+x^2 \right)^2}
    \end{align*}
    
    \item
    \begin{align*}
        f''(x) &= \left(f'(x) \right)'\\
        &= \left[ \left(-2x \right) \left(10+x^2 \right)^{-2}\right]'\\
        &= \left(-2x \right)'\left(10+x^2 \right)^{-2}
        + \left(-2x \right) \left[\left(10+x^2 \right)^{-2} \right]'\\
        &= -2\left(10+x^2 \right)^{-2}
        + \left(-2 \right) \left(-2x \right) \left(10+x^2 \right)^{-3} \left(2x \right)\\
        &= \frac{-2}{\left(10+x^2 \right)^{2}}
        + \frac{8x}{\left(10+x^2 \right)^{3}}\\
        &= \frac{-2\left(10+x^2 \right)+ 8x^2}{\left(10+x^2 \right)^{3}}\\
        &= \frac{6x^2- 20}{\left(10+x^2 \right)^{3}}
    \end{align*}
    
    \item
    \begin{align*}
        f'''(x) &= \left(f''(x) \right)'\\
        &= \left[\left(6x^2- 20 \right) \left(10+ x^2 \right)^{-3} \right]'\\
        &= \left(6x^2- 20 \right)'\left(10+x^2 \right)^{-3}
        + \left(6x^2- 20 \right) \left[\left(10+x^2 \right)^{-3} \right]'\\
        &= 12x \left(10+x^2 \right)^{-3}+ \left(-3 \right)\left(6x^2- 20 \right) \left(10+x^2 \right)^{-4} \left(2x \right)\\
        &= \frac{12x \left(10+ x^2 \right)- 6x\left(6x^2- 20 \right)}{\left(10+x^2 \right)^4}\\
        &= \frac{120x+ 12x^3+ 120x- 36x^3}{\left(10+x^2 \right)^4}\\
        &= \frac{-24x^3+ 240x}{\left(10+x^2 \right)^4}\\
        &= \frac{-24x\left(x^2- 10 \right)}{\left(10+x^2 \right)^4}\\
    \end{align*}
    
    
    
    \item
    \begin{align*}
        f''''(x) &= \left(f'''(x) \right)'\\
        &= \left[\left(-24x^3+ 240x \right) \left(10+ x^2 \right)^{-4} \right]'\\
        &= \left(-24x^3+ 240x \right)' \left(10+ x^2 \right)^{-4}+ \left(-24x^3+ 240x \right) \left[ \left(10+ x^2 \right)^{-4} \right]'\\
        &= \left(-72x^2+ 240 \right) \left(10+ x^2 \right)^{-4}+ \left(-4 \right) \left(-24x^3+ 240x \right) \left(10+ x^2 \right)^{-5} \left(2x \right)\\
        &= \frac{\left(-72x^2+ 240 \right) \left(10+ x^2 \right)- 8x\left(-24x^3+ 240x \right)}{\left(10+ x^2 \right)^5}\\
        &= \frac{-72x^4- 480x^2+ 2400+ 192x^4- 1920x^2}{\left(10+ x^2 \right)^5}\\
        &= \frac{120x^4- 2400x^2+ 2400}{\left(10+ x^2 \right)^5}\\
    \end{align*}
    
    
 \end{itemize}





\end{document}
