\documentclass[12pt]{article}
\usepackage{latexsym}
\usepackage{amsfonts}
\usepackage{amsmath}
\usepackage{pgfplots}


%%Formatting
\topmargin -.2in
\textheight 9.2in
\evensidemargin 0in
\oddsidemargin 0in
\textwidth 6.5in
\parskip .1in

\pgfplotsset{width=10cm,compat=1.9}



\pagestyle{empty}

\begin{document}

\hrule
\vspace{.2cm}

{\Large \noindent Math 112 Honors
\hfill
B\'ela Bajnok}

\vspace{.3cm}
\hrule

{\Large \noindent 
Exam 4
\hfill
Fall 2020}

\vspace{.3cm}
\hrule

\noindent NAME:  Quan Nguyen

\noindent \hrulefill\rule{0pt}{4pt}

\noindent Write and sign the full Honor Pledge here:

\vspace{2mm}

I affirm that I have upheld the highest principles of honesty and integrity in my academic work and have not witnessed a violation of the Honor Code. \par

Quan Nguyen

\vspace{8mm}

\noindent \hrulefill\rule{0pt}{4pt}

\noindent {\bf General instructions -- Please read!}

\begin{itemize}
  
\item The purpose of this exam is to give you an opportunity to explore a complex and challenging question, gain a fuller view of calculus and its applications, and develop some creative writing, problem solving, and research skills.

\item  {\bf All your assertions must be completely and fully justified.} At the same time, you should aim to be as concise as possible; avoid overly lengthy arguments and unnecessary components.  Your grade will be based on both mathematical accuracy and clarity of presentation.

\item 
Your finished exam should read as an article, consisting of complete sentences, thorough explanations, and exhibit correct grammar and punctuation.

\item I encourage you to prepare your exam using LaTeX.  However, you may use instead other typesetting programs that you like, and you may use hand-writing or hand-drawing for some parts of your exam.  In any case, {\bf the final version that you submit must be in PDF format.}  

\item It is acceptable (and even encouraged) to discuss the exams with other students in your class or with the PLA.  However,  {\bf  you must individually write up all parts of your exams.}

\item {\bf You may use the text, your notes, and your homework, but no other sources.}
 
\item You must write out a complete, honest, and detailed acknowledgment of all assistance you received and all resources you used (including other people) on all written work submitted for a grade.

\item Submit your exam to me by email at bbajnok@gettysburg.edu by the deadline announced in class.







\end{itemize}

\noindent {\bf Good luck!}


\newpage

\begin{center}

{\Large Probability -- Prove Your Ability} 

\end{center}

\vspace*{.2in}


\noindent  A certain university mathematics department offers, among others, a course on Combinatorics, a course on Probability, and a course on Statistics.  On a certain day, exams are given in each class, with a maximum possible score of 10 points in each; while in Combinatorics and Probability the achievable scores were all integers, in Statistics all values between 0 and 10 were possible.  The instructors gather the following information:

\begin{itemize}
  \item In the Combinatorics class, the scores were as follows:
  
$$1, 3, 4, 5, 5, 6, 6, 6, 7, 7, 7, 7, 7, 8, 8, 8, 8, 9, 9, 10.$$

\item In the Probability class, for each integer value of $n$ between 0 and 10, inclusive, the number of students who scored $n$ points was $(15n-n^2)/2$.

\item In the Statistics class, the number of students earning $x$ points (with $x$ between 0 and 10) could be approximated by the formula

$$63 \cdot \mathrm{e}^{-(x-7)^2/4}.$$

\end{itemize}

Answer the following questions for each of the three classes.  Use histograms, Riemann sums, or definite integrals as you see fit.


\begin{enumerate}

\item  How many students scored a 7?

\item  How many students scored a 7 or higher?

\item  How many students took the exam?

\item  What was the mode (the ``most frequent'' score)?

\item  What is the probability that a ``randomly'' selected student scored a 7 or higher?

\item  What was the mean (the ``average'' score)?

\item  What was the median (the ``middle'' score)?


\end{enumerate}

\newpage

\begin{center}
    \underline {\Large {My work} }
\end{center}

Some formulas and conventions that I use in this exam:
\begin{itemize}
    \item The mean score (average score):
    $$=\frac{\text{Total scores}}{\text{Total number of student}}$$
    \item The probability: 
    $$=\frac{\text{Number of student scored a 7 or higher}}{\text{Total number of student took exam}}$$
    \item All numbers are rounded to the nearest one.
    \item The total number of student taking the exam is equal to the total frequency of all scores from 0 to 10.
    \item All numbers with decimal are rounded to the nearest one.
\end{itemize}


\section*{Combinatorics class}
    \begin{center}
    \begin{tabular}{l | c*{10}r}
        Score &0 &1 &2 &3 &4 &5 &6 &7 &8 &9 &10 \\
        \hline
        Frequency &0 &1 &0 &1 &1 &2 &3 &5 &4 &2 &1
    \end{tabular}
    \end{center}


    \begin{enumerate}
        \item Number of students scored a 7: $5$.
        
        \item Number of students scored a 7 or higher: $12$.\par
        Explain: 5 students scored a 7; 4 students scored a 8; 2 students scored a 9; 1 students scored a 10.
        
        \item Number of students took the exam: 20. \par
        Explain: The number of students took the exam is equal to the total frequency of all scores from $1 \to 10$
        
        \item The most frequent score: $7$ \par
        Explain: The highest frequency is 5. The score with 5 in frequency is 7.
        
        \item Probability = $\frac{12}{20}= \frac{3}{5}$.
        
        \item The mean score = $\frac{1\cdot 1+3\cdot 1+4\cdot 1+5\cdot 2+6\cdot 3+7\cdot 5+8\cdot 4+9\cdot 2+10\cdot 1}{20}=
        \frac{131}{20} = 6.55$.
        
        \item There are 20 numbers in total, so the middle number will be a number that is in the middle of 10th and 11th number counting from score of $0\to 10$ \par
        The middle = $\frac{7+7}{2} = 7$
        
        \begin{center}
            \begin{tabular}{l | c|c|c|c|c|c|c|c|c|c |r}
                Score & 1 & 3 & 4 & 5 & \dots & 6 & 7 & 7 & 7 & \dots & 10 \\
                \hline
                &1st &2nd &3rd &4th &\dots &8th &9th &10th &11th &\dots &20th \\
            \end{tabular}
        \end{center}
        
        \vspace{0.5cm}

        %Column chart
        % Why still lose 10th column ???
        %\begin{tikzpicture}
        %\begin{axis}[
		%    xlabel=Score,
	    %    ylabel=Frequency,
	    %    xmin=0, xmax=10,
	    %    ymin=0, ymax=6,
	    %    enlargelimits=0.0,
	    %    legend columns=-1,
        %	ybar interval=1,
        %]
        %\addplot 
        %	coordinates {(0, 0) (1, 1) (2, 0) (3, 1) (4, 1) (5, 2) (6, 3) (7, 5) (8, 4) (9, 2) (10, 1)};
        %\end{axis}
        %\end{tikzpicture}

    \end{enumerate}
    
\section*{Probability class}
The general equation to calculate number of students (frequency) who scored n points: $$f(n)=\frac{15n-n^2}{2}$$ \par
    \begin{center}
    \begin{tabular}{l | c*{10}r}
        Score ($n$) & 0 & 1 & 2 & 3 & 4 & 5 & 6 & 7 & 8 & 9 & 10 \\
        \hline
        Frequency ($f(n)$)
        &0 &7 &13 &18 &22 &25 &27 &28 &28 &27 &25 \\
    \end{tabular}
\end{center}

% Notice: $\int_0^{10}\frac{15n-n^2}{2}\; \mathrm{d}n=208$ without $(-0.5 \to 0)$ and $(10 \to 10.5)$

    \begin{enumerate}
        \item Number of students scored a 7: 28.
        
        \item Number of students scored a 7 or higher: 108.\par
        $$\sum_{x=7}^{10}\frac{15x-x^2}{2}=108$$
        
        \item 
        Number of students took the exam: 220.\par
        $$\sum_{x=0}^{10}\frac{15x-x^2}{2}=220$$
            
        
        \item The most frequent score is 7 and 8. \par
        Explain: 28 students scored 7 and 28 students scored 8. 28 was also the most frequent score.
        

        \item Probability: $$=\;\frac{108}{220}\;=\;\frac{27}{55}$$
        
        \item The mean:
        $$=\frac{0+(1\cdot 7)+(2\cdot 13)+(3\cdot 18)+\cdots+(8\cdot 28)+(9\cdot 27)+(10\cdot 25)}{220}
        =\frac{25}{4}
        =6.25$$
        
        \item There are 220 students taking the exam, so the middle number will be a number that is in the middle of 110th and 111th number counting from score of $0 \to 10$\par
        The middle: $$=\frac{6+6}{2}=6$$
        
        \begin{center}
            \begin{tabular}{l |c|c|c|c|c|c|c|c|c|c|c| r}
                Score & 1 & 1 & 1 & 1 & \dots & 6 & 6 & 6 & 7 & \dots & 10 \\
                \hline
                &1st &2nd &3rd &4th &\dots &110th &111th &112th &113th &\dots &220th \\
            \end{tabular}
        \end{center}
        
    \end{enumerate}
    

\vspace{5cm}


%\begin{tikzpicture}
%\begin{axis}[
%    axis lines = left,
%    xlabel = $Score$,
%    ylabel = {$Frequency$},
%]
%
% How to move legend away from the line ???
%\addplot[
%    color=blue,
%    mark=square,
%    ]
%    coordinates {(0, 0) (1, 7) (2, 13) (3, 18) (4, 22) (5, 25) (6, 27) (7, 28) (8, 28) (9, 27) (10, 25)};
%    \addlegendentry{$\frac{15x-x^2}{2}$}

%\end{axis}
%\end{tikzpicture}

\newpage
\section*{Statistic class}
The general equation to calculate number of students (frequency) who scored $x$ points: $$f(x)=63e^{\frac{ -\left(x-7 \right)^2}{4}} $$ \par
\begin{center}
    \begin{tabular}{l | c*{10}r}
        Score ($x$) & 0 & 1 & 2 & 3 & 4 & 5 & 6 & 7 & 8 & 9 & 10 \\
        \hline
        Frequency &0 &0 &0 &1 &7 &24 &49 &62 &49 &24 &7 \\
    \end{tabular}
\end{center}

    \begin{enumerate}
        \item Number of students scored a 7: $$\int_{6.5}^{7.5} f(x)\;\mathrm{d}x = 61.7=62
        \qquad (\text{rounded to the nearest 1})$$. \par

        \item Number of students scored a 7 or higher (equal to sum number of students scored 7, 8, 9, and 10): 
        \begin{align*}
        &\int_{6.5}^{7.5} f(x)\;\mathrm{d}x 
        + \int_{7.5}^{8.5} f(x)\;\mathrm{d}x
        + \int_{8.5}^{9.5} f(x)\;\mathrm{d}x
        + \int_{9.5}^{10} f(x)\;\mathrm{d}x \\
        =\quad& \int_{6.5}^{10} f(x)\;\mathrm{d}x \\
        =\quad& 138.7 \\
        =\quad& 139 \qquad (\text{rounded to the nearest 1})
        \end{align*}

        \item Number of students took the exam (equal to total number of students scored from $0\to 10$):
        \begin{align*}
            &\int_{0}^{0.5} f(x)\;\mathrm{d}x 
            + \int_{0.5}^{1.5} f(x)\;\mathrm{d}x 
            +\cdots
            + \int_{8.5}^{9.5} f(x)\;\mathrm{d}x
            + \int_{9.5}^{10} f(x)\;\mathrm{d}x \\
            =\quad&\int_{0}^{10} f(x)\;\mathrm{d}x \\
            =\quad& 219.5 \\
            =\quad& 220 \qquad (\text{rounded to the nearest 1})
        \end{align*}
        
        
        \item The most frequent score: 7. Because the highest frequency was 62 and 62 students scored 7.
        
        \item Probability:
        $$\left(\int_{6.5}^{10} f(x)\;\mathrm{d}x \right) 
        \;\div 
        \left(\int_{0}^{10} f(x)\;\mathrm{d}x \right)
        = \frac{139}{220}$$

        
        \item The mean:
        \begin{align*}
            &=\left( \int_0^{10}x\cdot f(x) \;\mathrm{d} x\right)
            \div
            \left(\int_0^{10} f(x) \;\mathrm{d}x \right) \\
            &= \frac{1}{220} \int_0^{10}x\cdot f(x) \;\mathrm{d}x \\
            &= 6.93
        \end{align*}
        

        \item There are 220 students taking the exam, so the middle number will be a number that is in the middle of 110th and 111th number counting from score of $0 \to 10$. \par
        The middle $$=\frac{7+7}{2}= 7 $$
        
        \begin{center}
            \begin{tabular}{l | c|c|c|c|c|c|c|c|c|c |r}
                Score & 3 & 4 & 4 & 4 & 4 & \dots & 7 & 7 & \dots & 10 \\
                \hline
                &1st &2nd &3rd &4th &5th &\dots &110th &111th &\dots & 220th \\
            \end{tabular}
        \end{center}
        
    \end{enumerate}


\end{document}
